\newpage
\section*{A2.4}
\textbf{a)}\\
\begin{center}
\begin{rndtable}{|c|c|c|}
\hline
 \rowcolor{gray}
 \textbf{Type} & \textbf{Range} & \textbf{Step}\\

 \hline
 0.0 & 0 & 0 \\

 \hline
 0.1 & -0.9 to 0.9 & 0.1\\

 \hline
 1.0 & -9 to 9 & 1\\

 \hline
 1.1 & -9.9 to 9.9 & 0.1\\

 \hline
\end{rndtable}
\end{center}
\begin{center}
  $0.0 \preceq 1.0 \preceq 1.1$\\
  $0.0 \preceq 0.1 \preceq 1.1$\\
\end{center}
From the table above, it has been deduced that one would need to define the following conditions:
\begin{center}
  $m \leq p$ $\wedge$ $n \leq q$
\end{center}
In order to ensure $m.n \preceq p.q$. That is, a value of type $m.n$ can always be converted to a value of type $p.q$ without overflow and without loss of precision.\\

\textbf{b)}\\
A type $a.b$ such that $a.b \preceq p.q$ for any $p$ and $q$ is: \textbf{type 0.0}.\\

\textbf{c)}\\
A type $a.b$ such that $m.n \preceq a.b$ for any $m$ and $n$ does not exist, because once we fix $a$ and $b$, we can always find $m$ and $n$ such that $m > a$ $\wedge$ $n > b$ contradicting the conditions in $a)$.\\

\textbf{d)}\\
$$
  y:= (p.q)x = (p.q)(m.n)
$$
\\
The downcast operation must perform the following checks:
\begin{itemize}
  \item \textbf{Overflow}: We need to determine the range for $y:\; p.q$ and then check whether or not the value of x lies within that range.
  \item \textbf{Precision}: We need to determine the number of digits, $i$, after the decimal point of $x$ and check whether or not $i \leq q$.
\end{itemize}
So in short, x must lie within the range of $y$ and have less or equal significant digits than what is allowed by $y$.
\newpage
\textbf{e)}\\
In e), f) and g) we have:
\begin{align*}
       x:& (m.n)\\
       y:& (p.q)
\end{align*}
The ranges for $x$ and $y$:
\begin{align*}
  x:\quad & - 9_09_1...9_m {\Large \textbf{.}} 9_09_1...9_n \texttt{ to } 9_09_1...9_m{\Large \textbf{.}}9_09_1...9_n\\
    &\hspace{25mm} or \\
  & - (10^m - \frac{1}{10^n} ) \texttt{ to }  (10^m - \frac{1}{10^n} )\\
  \hspace{1cm}\\
  y:\quad & - 9_09_1...9_p {\Large \textbf{.}} 9_09_1...9_q \texttt{ to } 9_09_1...9_p {\Large \textbf{.}} 9_09_1...9_q\\
  &\hspace{25mm} or \\
  & - (10^p - \frac{1}{10^q} ) \texttt{ to }  (10^p - \frac{1}{10^q} )
\end{align*}
\\
We assign the most precise type for $z$ in the expression $z= x+y$.\\
We can at max overflow with 1 for the left-hand-side of the type declaration, which is evident when adding the maximum representable numbers of $x$ and $y$, $MAX(x)$ and $MAX(y)$:
\begin{align*}
  z &= MAX(x) + MAX(y)\\
  z &=  9_09_1...9_m{\Large \textbf{.}}9_09_1...9_n + 9_09_1...9_p {\Large \textbf{.}} 9_09_1...9_q
\end{align*}
Depending on which \{$m$ and $p$\} and \{$n$ and $q$\} is greatest, we then get:
\begin{center}
  $z$: $max(m,p)+1.max(n,q)$
\end{center}
(The same holds when looking at the minimum representable numbers of $x$ and $y$)\\
\\
\textbf{f)}\\
The most precise type I can give for $k$ in $k= x*y$.
\begin{center}
  $k$: $(m+p).(n+q)$
\end{center}
\underline{Explanation of $k's$ type}:\\
We can derive $k's$ type by using $l$ in the following:
\begin{align*}
  l &= MAX(x) * MAX(y)\\
    &= \Big(10^m - \frac{1}{10^n}\Big) * \Big(10^p - \frac{1}{10^q}\Big)\\
    &= 10^m10^p - 10^m\frac{1}{10^q} - 10^p\frac{1}{10^n} +  \frac{1}{10^n}\frac{1}{10^q}\\
    &= 10^{m+p} - 10^m\frac{1}{10^q} - 10^p\frac{1}{10^n} +  \frac{1}{10^{n+q}}\\
    &= 10^{m+p} - 10^{m-q} - 10^{p-n} + 10^{-(n+q)}
\end{align*}
Based on the first and last term of the last equation, we can see that the resulting type of $l$ must be $(m+p).(n+q)$. This means we must verify:
\begin{align*}
  l \quad &< \quad MAX( (m+p).(n+q))\\
  10^{m+p} - 10^{m-q} - 10^{p-n} + 10^{-(n+q)} \quad  &< \quad 10^{m+p} - \frac{1}{10^{n+q}} \\
  - 10^{m-q} - 10^{p-n} + 10^{-(n+q)} \quad  &< \quad - 10^{-(n+q)} \\
  - 10^{m-q} - 10^{p-n} + 2 \cdot 10^{-(n+q)} \quad  &< \quad 0\\
  - 10^{m-q} - 10^{p-n} \quad  &< \quad  - 2 \cdot 10^{-(n+q)}\\
  - 10^{m+n} - 10^{p+q} \quad  &< \quad  - 2 \cdot 10^{-(n+q)+(n+q)}\\
  - 10^{m+n} - 10^{p+q} \quad  &< \quad  - 2
\end{align*}
Considering $m=n=p=q=0$ is pointless, so we see that for any $m,n,p,q$ greater than 0, the above holds. In a similar way, we can arrive at the same truth when considering the minimum representable numbers, instead of the maximum as we did above. Thus we have that the most precise type for $k$: $(m+p).(n+q)$.\\
\\
\textbf{g)}\\
Now I will define which values of $m,\ n,\ p$ and $q$ for which this assignment will be type safe:
$$
  x = x + y
$$
For an arbitrary $a$ and $b$:
\begin{center}
  $(x:\ m.n = a.b \qquad  \wedge \qquad y:\ p.q = 0.0)$
\end{center}

\underline{Modification}:\\
We could do the following modifications in order to make the assignment type safe:
\begin{enumerate}
  \item Before the assignment, determine the type of $x: m.n$ and $y: n.q$
  \item Before the assignment, upcast $x$: $(max(m,p)+1.max(n,q))\ x$.
  \item Now we can safely do the assignment
\end{enumerate}
\textcolor{red}{\textbf{Modification: (left the previous answer as is for comparison)}}\\
The only sensible way to circumvent this without changing the type of x, is by casting the following way:
$$
x = (m.n)(x + (m.n)y)
$$
However, this approach only works, when we are actually able to downcast y.
